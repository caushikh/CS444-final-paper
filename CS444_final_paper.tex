\documentclass[letterpaper,10pt,titlepage]{article}

\usepackage{graphicx}                                        
\usepackage{amssymb}                                         
\usepackage{amsmath}                                         
\usepackage{amsthm}                                          

\usepackage{alltt}                                           
\usepackage{float}
\usepackage{color}
\usepackage{url}

\usepackage{hyperref}
\usepackage[vertfit]{breakurl}

\usepackage{ragged2e}
\edef\UrlBreaks{\do\-\UrlBreaks}

\usepackage{balance}
\usepackage[TABBOTCAP, tight]{subfigure}
\usepackage{enumitem}
\usepackage{pstricks, pst-node}

\usepackage{geometry}
\geometry{textheight=8.5in, textwidth=6in}

%random comment

\newcommand{\cred}[1]{{\color{red}#1}}
\newcommand{\cblue}[1]{{\color{blue}#1}}

\usepackage{hyperref}
\usepackage{geometry}

\def\name{Hari Caushik}

%pull in the necessary preamble matter for pygments output
\input{pygments.tex}

%% The following metadata will show up in the PDF properties
\hypersetup{
  colorlinks = true,
  linkcolor = black,
  urlcolor = black,
  citecolor = black,
  urlbordercolor = black,
  runbordercolor = black,
  pdfauthor = {\name},
  pdfkeywords = {CS444 ``Operating Systems II'' Final Paper},
  pdftitle = {CS 444 Final Paper},
  pdfsubject = {CS 444 Final Paper},
  pdfpagemode = UseNone
}
\urlstyle{same}

\title{Operating System Feature Comparison}
\date{December 11, 2014}
\author{Hari Caushik}

\begin{document}

\maketitle
\thispagestyle{empty}
\newpage

\tableofcontents
\newpage

\section{Processes and Threads}
\subsection{Similarities}
The information contained in the process containers for both Linux and Windows
are very similar. Specifically, all processes contain a private virtual 
address space, program code, sets of resources like opened files and signals,
and multiple threads of execution. 
\subsection{Differences}
One fundamental difference between Linux and Windows is in process creation. 
In Linux, new processes are created by invoking two different system calls:
fork(), to create a duplicate of the current process, and exec(), to load an
executable file into the new process's address space and then execute it. 
In Windows, a call to one of the process creation functions like CreateProcess
loads an executable image file, or an .exe file, and creates the Windows
executive process object all in one. 
\subsection{Why Similarities and Differences Exist?}
\section{Memory Management}
\subsection{Similarities}
\subsection{Differences}
\subsection{Why Similarities and Differences Exist?}
\section{I/O}
\subsection{Similarities}
\subsection{Differences}
\subsection{Why Similarities and Differences Exist?}
%input the pygmentized output of mt19937ar.c, using a (hopefully) unique name
%this file only exists at compile time. Feel free to change that.
%\input{__mt19937ar.c.tex}
\nocite{*}
\RaggedRight
\bibliographystyle{plain-annote}
\bibliography{CS444_final_paper}
\end{document}
