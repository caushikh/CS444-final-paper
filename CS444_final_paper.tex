\documentclass[letterpaper,10pt,titlepage]{article}

\usepackage{graphicx}                                        
\usepackage{amssymb}                                         
\usepackage{amsmath}                                         
\usepackage{amsthm}                                          

\usepackage{alltt}                                           
\usepackage{float}
\usepackage{color}
\usepackage{url}

\usepackage{hyperref}
\usepackage[vertfit]{breakurl}

\usepackage{ragged2e}
\edef\UrlBreaks{\do\-\UrlBreaks}

\usepackage{balance}
\usepackage[TABBOTCAP, tight]{subfigure}
\usepackage{enumitem}
\usepackage{pstricks, pst-node}

\usepackage{geometry}
\geometry{textheight=8.5in, textwidth=6in}

%random comment

\newcommand{\cred}[1]{{\color{red}#1}}
\newcommand{\cblue}[1]{{\color{blue}#1}}

\usepackage{hyperref}
\usepackage{geometry}

\def\name{Hari Caushik}

%pull in the necessary preamble matter for pygments output
\usepackage{fancyvrb}
\usepackage{color}
\usepackage[latin1]{inputenc}


\makeatletter
\def\PY@reset{\let\PY@it=\relax \let\PY@bf=\relax%
    \let\PY@ul=\relax \let\PY@tc=\relax%
    \let\PY@bc=\relax \let\PY@ff=\relax}
\def\PY@tok#1{\csname PY@tok@#1\endcsname}
\def\PY@toks#1+{\ifx\relax#1\empty\else%
    \PY@tok{#1}\expandafter\PY@toks\fi}
\def\PY@do#1{\PY@bc{\PY@tc{\PY@ul{%
    \PY@it{\PY@bf{\PY@ff{#1}}}}}}}
\def\PY#1#2{\PY@reset\PY@toks#1+\relax+\PY@do{#2}}

\expandafter\def\csname PY@tok@gd\endcsname{\def\PY@tc##1{\textcolor[rgb]{0.63,0.00,0.00}{##1}}}
\expandafter\def\csname PY@tok@gu\endcsname{\let\PY@bf=\textbf\def\PY@tc##1{\textcolor[rgb]{0.50,0.00,0.50}{##1}}}
\expandafter\def\csname PY@tok@gt\endcsname{\def\PY@tc##1{\textcolor[rgb]{0.00,0.25,0.82}{##1}}}
\expandafter\def\csname PY@tok@gs\endcsname{\let\PY@bf=\textbf}
\expandafter\def\csname PY@tok@gr\endcsname{\def\PY@tc##1{\textcolor[rgb]{1.00,0.00,0.00}{##1}}}
\expandafter\def\csname PY@tok@cm\endcsname{\let\PY@it=\textit\def\PY@tc##1{\textcolor[rgb]{0.25,0.50,0.50}{##1}}}
\expandafter\def\csname PY@tok@vg\endcsname{\def\PY@tc##1{\textcolor[rgb]{0.10,0.09,0.49}{##1}}}
\expandafter\def\csname PY@tok@m\endcsname{\def\PY@tc##1{\textcolor[rgb]{0.40,0.40,0.40}{##1}}}
\expandafter\def\csname PY@tok@mh\endcsname{\def\PY@tc##1{\textcolor[rgb]{0.40,0.40,0.40}{##1}}}
\expandafter\def\csname PY@tok@go\endcsname{\def\PY@tc##1{\textcolor[rgb]{0.50,0.50,0.50}{##1}}}
\expandafter\def\csname PY@tok@ge\endcsname{\let\PY@it=\textit}
\expandafter\def\csname PY@tok@vc\endcsname{\def\PY@tc##1{\textcolor[rgb]{0.10,0.09,0.49}{##1}}}
\expandafter\def\csname PY@tok@il\endcsname{\def\PY@tc##1{\textcolor[rgb]{0.40,0.40,0.40}{##1}}}
\expandafter\def\csname PY@tok@cs\endcsname{\let\PY@it=\textit\def\PY@tc##1{\textcolor[rgb]{0.25,0.50,0.50}{##1}}}
\expandafter\def\csname PY@tok@cp\endcsname{\def\PY@tc##1{\textcolor[rgb]{0.74,0.48,0.00}{##1}}}
\expandafter\def\csname PY@tok@gi\endcsname{\def\PY@tc##1{\textcolor[rgb]{0.00,0.63,0.00}{##1}}}
\expandafter\def\csname PY@tok@gh\endcsname{\let\PY@bf=\textbf\def\PY@tc##1{\textcolor[rgb]{0.00,0.00,0.50}{##1}}}
\expandafter\def\csname PY@tok@ni\endcsname{\let\PY@bf=\textbf\def\PY@tc##1{\textcolor[rgb]{0.60,0.60,0.60}{##1}}}
\expandafter\def\csname PY@tok@nl\endcsname{\def\PY@tc##1{\textcolor[rgb]{0.63,0.63,0.00}{##1}}}
\expandafter\def\csname PY@tok@nn\endcsname{\let\PY@bf=\textbf\def\PY@tc##1{\textcolor[rgb]{0.00,0.00,1.00}{##1}}}
\expandafter\def\csname PY@tok@no\endcsname{\def\PY@tc##1{\textcolor[rgb]{0.53,0.00,0.00}{##1}}}
\expandafter\def\csname PY@tok@na\endcsname{\def\PY@tc##1{\textcolor[rgb]{0.49,0.56,0.16}{##1}}}
\expandafter\def\csname PY@tok@nb\endcsname{\def\PY@tc##1{\textcolor[rgb]{0.00,0.50,0.00}{##1}}}
\expandafter\def\csname PY@tok@nc\endcsname{\let\PY@bf=\textbf\def\PY@tc##1{\textcolor[rgb]{0.00,0.00,1.00}{##1}}}
\expandafter\def\csname PY@tok@nd\endcsname{\def\PY@tc##1{\textcolor[rgb]{0.67,0.13,1.00}{##1}}}
\expandafter\def\csname PY@tok@ne\endcsname{\let\PY@bf=\textbf\def\PY@tc##1{\textcolor[rgb]{0.82,0.25,0.23}{##1}}}
\expandafter\def\csname PY@tok@nf\endcsname{\def\PY@tc##1{\textcolor[rgb]{0.00,0.00,1.00}{##1}}}
\expandafter\def\csname PY@tok@si\endcsname{\let\PY@bf=\textbf\def\PY@tc##1{\textcolor[rgb]{0.73,0.40,0.53}{##1}}}
\expandafter\def\csname PY@tok@s2\endcsname{\def\PY@tc##1{\textcolor[rgb]{0.73,0.13,0.13}{##1}}}
\expandafter\def\csname PY@tok@vi\endcsname{\def\PY@tc##1{\textcolor[rgb]{0.10,0.09,0.49}{##1}}}
\expandafter\def\csname PY@tok@nt\endcsname{\let\PY@bf=\textbf\def\PY@tc##1{\textcolor[rgb]{0.00,0.50,0.00}{##1}}}
\expandafter\def\csname PY@tok@nv\endcsname{\def\PY@tc##1{\textcolor[rgb]{0.10,0.09,0.49}{##1}}}
\expandafter\def\csname PY@tok@s1\endcsname{\def\PY@tc##1{\textcolor[rgb]{0.73,0.13,0.13}{##1}}}
\expandafter\def\csname PY@tok@sh\endcsname{\def\PY@tc##1{\textcolor[rgb]{0.73,0.13,0.13}{##1}}}
\expandafter\def\csname PY@tok@sc\endcsname{\def\PY@tc##1{\textcolor[rgb]{0.73,0.13,0.13}{##1}}}
\expandafter\def\csname PY@tok@sx\endcsname{\def\PY@tc##1{\textcolor[rgb]{0.00,0.50,0.00}{##1}}}
\expandafter\def\csname PY@tok@bp\endcsname{\def\PY@tc##1{\textcolor[rgb]{0.00,0.50,0.00}{##1}}}
\expandafter\def\csname PY@tok@c1\endcsname{\let\PY@it=\textit\def\PY@tc##1{\textcolor[rgb]{0.25,0.50,0.50}{##1}}}
\expandafter\def\csname PY@tok@kc\endcsname{\let\PY@bf=\textbf\def\PY@tc##1{\textcolor[rgb]{0.00,0.50,0.00}{##1}}}
\expandafter\def\csname PY@tok@c\endcsname{\let\PY@it=\textit\def\PY@tc##1{\textcolor[rgb]{0.25,0.50,0.50}{##1}}}
\expandafter\def\csname PY@tok@mf\endcsname{\def\PY@tc##1{\textcolor[rgb]{0.40,0.40,0.40}{##1}}}
\expandafter\def\csname PY@tok@err\endcsname{\def\PY@bc##1{\setlength{\fboxsep}{0pt}\fcolorbox[rgb]{1.00,0.00,0.00}{1,1,1}{\strut ##1}}}
\expandafter\def\csname PY@tok@kd\endcsname{\let\PY@bf=\textbf\def\PY@tc##1{\textcolor[rgb]{0.00,0.50,0.00}{##1}}}
\expandafter\def\csname PY@tok@ss\endcsname{\def\PY@tc##1{\textcolor[rgb]{0.10,0.09,0.49}{##1}}}
\expandafter\def\csname PY@tok@sr\endcsname{\def\PY@tc##1{\textcolor[rgb]{0.73,0.40,0.53}{##1}}}
\expandafter\def\csname PY@tok@mo\endcsname{\def\PY@tc##1{\textcolor[rgb]{0.40,0.40,0.40}{##1}}}
\expandafter\def\csname PY@tok@kn\endcsname{\let\PY@bf=\textbf\def\PY@tc##1{\textcolor[rgb]{0.00,0.50,0.00}{##1}}}
\expandafter\def\csname PY@tok@mi\endcsname{\def\PY@tc##1{\textcolor[rgb]{0.40,0.40,0.40}{##1}}}
\expandafter\def\csname PY@tok@gp\endcsname{\let\PY@bf=\textbf\def\PY@tc##1{\textcolor[rgb]{0.00,0.00,0.50}{##1}}}
\expandafter\def\csname PY@tok@o\endcsname{\def\PY@tc##1{\textcolor[rgb]{0.40,0.40,0.40}{##1}}}
\expandafter\def\csname PY@tok@kr\endcsname{\let\PY@bf=\textbf\def\PY@tc##1{\textcolor[rgb]{0.00,0.50,0.00}{##1}}}
\expandafter\def\csname PY@tok@s\endcsname{\def\PY@tc##1{\textcolor[rgb]{0.73,0.13,0.13}{##1}}}
\expandafter\def\csname PY@tok@kp\endcsname{\def\PY@tc##1{\textcolor[rgb]{0.00,0.50,0.00}{##1}}}
\expandafter\def\csname PY@tok@w\endcsname{\def\PY@tc##1{\textcolor[rgb]{0.73,0.73,0.73}{##1}}}
\expandafter\def\csname PY@tok@kt\endcsname{\def\PY@tc##1{\textcolor[rgb]{0.69,0.00,0.25}{##1}}}
\expandafter\def\csname PY@tok@ow\endcsname{\let\PY@bf=\textbf\def\PY@tc##1{\textcolor[rgb]{0.67,0.13,1.00}{##1}}}
\expandafter\def\csname PY@tok@sb\endcsname{\def\PY@tc##1{\textcolor[rgb]{0.73,0.13,0.13}{##1}}}
\expandafter\def\csname PY@tok@k\endcsname{\let\PY@bf=\textbf\def\PY@tc##1{\textcolor[rgb]{0.00,0.50,0.00}{##1}}}
\expandafter\def\csname PY@tok@se\endcsname{\let\PY@bf=\textbf\def\PY@tc##1{\textcolor[rgb]{0.73,0.40,0.13}{##1}}}
\expandafter\def\csname PY@tok@sd\endcsname{\let\PY@it=\textit\def\PY@tc##1{\textcolor[rgb]{0.73,0.13,0.13}{##1}}}

\def\PYZbs{\char`\\}
\def\PYZus{\char`\_}
\def\PYZob{\char`\{}
\def\PYZcb{\char`\}}
\def\PYZca{\char`\^}
\def\PYZam{\char`\&}
\def\PYZlt{\char`\<}
\def\PYZgt{\char`\>}
\def\PYZsh{\char`\#}
\def\PYZpc{\char`\%}
\def\PYZdl{\char`\$}
\def\PYZti{\char`\~}
% for compatibility with earlier versions
\def\PYZat{@}
\def\PYZlb{[}
\def\PYZrb{]}
\makeatother

%% The following metadata will show up in the PDF properties
\hypersetup{
  colorlinks = true,
  linkcolor = black,
  urlcolor = black,
  citecolor = black,
  urlbordercolor = black,
  runbordercolor = black,
  pdfauthor = {\name},
  pdfkeywords = {CS444 ``Operating Systems II'' Final Paper},
  pdftitle = {CS 444 Final Paper},
  pdfsubject = {CS 444 Final Paper},
  pdfpagemode = UseNone
}
\urlstyle{same}

\title{Operating System Feature Comparison}
\date{December 11, 2014}
\author{Hari Caushik}

\begin{document}

\maketitle
\thispagestyle{empty}
\newpage

\tableofcontents
\newpage

\section{Processes and Threads}
\subsection{Similarities}
The information contained in the process containers for both Linux and Windows
are very similar. Specifically, all processes contain a private virtual 
address space, program code, sets of resources like opened files and signals,
and multiple threads of execution. The process descriptors contain references
to machine registers, CPU time accounting, kernel stacks, and file-descriptor
tables to locate system calls. In Windows, this is the Executive Process Block
and in Linux, processes are represented by task\_struct structures. The 
Executive Process Block of each process contains a Process Environment Block 
that is accessible in user space. It includes information about the executable
image, process heap, and thread storage. The task\_struct also contains 
information accessible in user space like the heap, stack, and global
variables.
\\
\linebreak
Both Linux and Windows implement user mode and kernel mode. In both operating
systems, a thread can invoke a system call that causes the kernel to execute
kernel code on behalf of the thread. The thread contains a separate kernel 
mode stack and kernel mode program counter for execution in kernel mode. These
are saved and restarted when switching from kernel mode to user mode and vice
versa. The Windows process address space includes a process environment block
and thread environment block that includes information that is accessible in
user mode. This includes the information described above in the previous
paragraph. The actual process executive block itself is only accessible in 
kernel mode and can only be manipulated by well defined system calls. In Linux,
some parts of the task\_struct like the heap, stack, and global variables are
accessible in user space, but other information like the task state is only
accessible in kernel space. 
\\
\linebreak
Both Linux and Windows provide support for background processes. In Windows,
these are called Windows Services and are managed by the Service Control 
Manager. In Linux, background processes are called daemons and are created 
when a process is forked and its parent terminates. The new process then calls
setsid() to free itself from the controlling terminal. Windows Services and 
daemons are both usually created at system startup. Background processes in
Windows and Linux usually are usually long-lived and run until the system 
is shut down. They usually carry out specific tasks like sshd and httpd, the
two Linux daemons that manage logins from remote hosts using the ssh protocol
and serve web pages respectively. An example of a Windows Services is 
the plug and play manager that automatically recognizes devices and installs
the necessary driver software.
\\
\linebreak
Both operating systems provide support for threads that only run in kernel 
context. In Linux, these are known as kernel threads. In Windows, these are 
known as system threads. System threads can be created with the system call
PsCreateSystemThread. Linux kernel threads are usually created at system 
startup. In Linux, these threads are not represented by a separate task\_struct
and do not have a separate process context. Likewise, in Windows, these
System threads do not have a Thread Environment Block (TEB). They lack a user
process address space. In Windows, kernel mode system threads are often 
created at system startup to perform operations like writing dirty pages to
page files or swapping processes in and out of memory. In Linux, kernel 
threads execute tasks like pdflush to write dirty pages to page files and 
ksoftirqd to handle soft interrupts.
\subsection{Differences}

One fundamental difference between Linux and Windows is in process creation. 
In Linux, new processes are created by invoking two different system calls:
fork(), to create a duplicate of the current process, and exec(), to load an
executable file into the new process's address space and then execute it. 
In Windows, a call to one of the process creation functions like CreateProcess
loads an executable image file, or an .exe file, and creates the Windows
executive process object all in one. In Linux, this allows a forked process
to do some work before it loads a new executable image. In Windows, one can
duplicate a process by specifying the executable module argument in 
CreateProcess as NULL. 
\\
\linebreak
The following is a short snippet of code demonstrating the use of fork() and
then the call to execv() to create the new process with the new executable
file image in Linux.

\input{__execeg.c.tex}

Here is an example of the CreateProcess function in Windows. Notice that the
name of the module to be executed is an argument to the function whereas in
Linux, a separate exec call had to be made.

\input{__wincreateproc.c.tex}

A significant difference between Windows and Linux is the implementation of 
threads. In Linux, threads are implemented as task\_structs just like 
standard processes and simply share resources with other processes or threads.
In Windows, processes contain one or more threads and only those threads are
schedulable by the scheduler. Windows dedicates a special data structure to 
threads called the executive thread block. There is nothing different in the
way that threads are represented in Linux except for the fact that they share
part of the address space as their parent. Their process id is also the same
as their parent. In Windows, every process contains one or more threads
and cannot execute without these threads. Processes are designed to be 
resource managers and threads are designed to be the units of execution.
\\
\linebreak
In Linux, processes are schedulable and have schedulable states like running,
zombie, and interruptiple. In Windows only threads have schedulable states and
processes are not schedulable. Processes in Windows only exist to serve as a
container for resources while in Linux, processes do that as well as execute 
code. In Windows, these states are ready, standby, running, waiting, 
transition, terminated, and initialized. The ready state is for threads 
that are waiting to execute. The running and standby states are for threads
that are executing and are selected next to run. The wait state is for 
threads that have been preempted for specific reasons. The transition state 
are for threads that are ready for execution but whose kernel stack has been
swapped out to disk. Threads in the terminated state are done executing
and the initialized state signifies that a thread has just been created. 
\\
\linebreak
Windows also contains entities called jobs and fibers that are not supported 
by Linux. Jobs are collections of processes that can be managed as a unit.
Fibers are essentially threads that are scheduled in user space instead of
by the kernel's scheduling algorithm. A Windows process can belong to only
one job object. A job object can specify the CPU time allocated to threads
belonging to a set of processes. Job objects are created and opened with the
API CreateJobObject and OpenJobObject. Fibers are switched with much less
overhead than switching between threads. It avoids the need to enter and 
exit the kernel.

\subsection{Why Similarities and Differences Exist?}
The process address space is present and is similar in both Windows and Linux
because both operating systems need to provide structure to running programs
and provide an abstraction that individual programs monopolize the processor.
Application programs should not have to worry about details like managing 
resources it needs. This improves programmer productivity and allows for the
creation of a number of different applications. This has enabled a tremendous 
amount of software to be engineered for both Windows and Linux. 
\\
\linebreak
Both operating systems support user mode and kernel mode because it protects
the operating system from being damaged in a fatal way. In addition, since 
both operating systems support multiple users and support multitasking, 
failures in one application will not cause the entire system to crash. 
Protecting the operating system from modification by application programs 
ensures that the system will continue to run reliably. In addition, it 
relieves application programs from having to manage the system and worry that
anything it will do will affect the system. It is a fundamental abstraction of 
operating systems that essentially all operating systems will have.
\\
\linebreak
Background processes are supported by both operating systems because they can
support services to more than one user and more than one process or 
application that these applications or users do not need to be concerned with.
Certain services like a secure shell server or http server are not 
interactive, but are used often enough by many different applications and 
users that it is necessary to have these processes in memory. Both operating
systems have services like these that they need to support non-interactively.
\\
\linebreak
System threads or kernel threads are useful for both operating systems 
because both have tasks that must be frequently executed and that can only be
executed in kernel mode. The frequency with which dirty pages are written 
back to disk, for example,  warrants a separate thread or execution unit that
continuously executes in kernel space. 
\\
\linebreak
Linux provides 2 calls to create a new process and then load a new executable 
image because one of Linux's main design principles is for each function to 
do one thing and do it well. It provides one function to create the process
and it provides another function to load a new executable image. The design 
is simple and modular. Changing the functionality in one of these functions
will not affect the other. In Windows, simplicity is not necessarily a 
design goal and so it does not see a need to separate these two 
functionalities. Creating new processes and specifying the new 
executable image with an argument avoids the redundancy of making two calls
to do this. In addition, Windows does not anticipate that this 
functionality will necessarily change and so there is no need for the 
modularity.
\\
\linebreak
The implementation of threads in Linux is based on providing a mechanism to 
share resources between processes while in Windows, threads exist to provide
quicker execution. Thus they are represented by a separate data structure that 
is more lightweight. Linux threads share the same heap and global variables.
On the contrary processes have completely different address spaces. Windows
threads are the schedulable execution units. They do not contain any 
resources themselves, but contain a reference to the executive process block
that created them. They can thus run faster without the overhead of processes
and can reference the executive process block whenever they need them. 
\\
\linebreak
Processes in Windows do not have schedulable states because they are not 
meant to be units of execution. Windows also has more states for its threads
because the dispatcher, or scheduler, is more complex. Threads are more 
lightweight in Windows and thus their can be many more threads that are 
created at any point in time. More threads to manage naturally leads to
more schedulable states and more complex scheduling algorithms. 
\\
\linebreak
Jobs exist in Windows in order to constrain the threads they create in 
certain ways. For example, they can be used to control access to a certain
resource or prevent threads from accessing too many system objects. Because
a process is part of a job, all threads that are created from that process
will also be in that job and can be controlled collectively. Fibers exist
to provide an execution unit that can be scheduled and run without the
overhead of threads. Linux does not want all these different execution
units or execution managers because the main design goal is simplicity. 
It wants as few execution units as possible and keep life simple for 
application programs.
\section{Memory Management}
\subsection{Similarities}
In both Linux and Windows, every process has its own virtual address space.
In Windows, the process's virtual address space is managed by a component
of the operating system called the memory manager. The subset of a process's
virtual address space in Windows that is in memory is called a working set.
In both Linux and Windows, the virtual address space provided to a process
can be smaller than the amount of physical memory available. Thus, both 
operating systems make use of paging to provide this illusion.
\\
\linebreak
Both Windows and Linux support copy-on-write paging. When an attempt is made
to modify pages that start out originally as read-only pages, the memory 
manager makes a private writable copy of the page and updates the page table 
to reference the new private page. The new page will then be paged out rather
than the original page. This is also called lazy evaluation. The actual copy
of a page will only be made by when there is an attempt to modify it. In 
Linux, one of the main ways in which copy-on-write is used is when making the
fork() system call. The process address space is not copied straight away 
because often a call to exec() is called soon after and the process address
space would have to be changed again. Instead, copy-on-write defers the 
copying of the address space until a change to the address space is actually 
attempted.
\subsection{Differences}
Windows provides support for shared memory objects directly by memory 
mapped files that multiple processes can open in their virtual address space.
On the contrary, in Linux one process must allocate a shared memory segment and
then one or more other processes attach it to their address space. The 
mechanism by which Linux provides shared memory objects is different than 
simple memory mapping a file. It must do the additional step of assigning the
memory mapped file as a shared memory object that other processes can attach
to their address space.
\\
\linebreak
Windows and Linux processes have different amounts of their virtual address
space dedicated to kernel virtual memory. Windows dedicates 2 GB of each 4 GB
process address space for kernel virtual memory. Linux dedicates 1 GB of each
4 GB process address space for kernel virtual memory. A consequence of this 
is that Windows processes have less private virtual memory. In return, they 
can execute system calls faster because the kernel can execute more code 
within the process's kernel virtual memory and no context switch would need to
be made. Linux provides less kernel virtual memory in a process's address 
space, causing more context switches to take place. 
\\ 
\linebreak
Linux manages 3 levels of page tables to map virtual addresses into physical
addresses. These are the Page General Directory (PGD), Page Middle Directory
(PMD), and the Page Table Entry (PTE). Windows has 2 levels of page tables. 
These are the Page Directory and the Page Tables.
\subsection{Why Similarities and Differences Exist?}
Both Windows and Linux provide a virtual address space for its processes
because they are a useful abstraction for any sort of operating system 
design. Every process can have its own private memory and relieves application
programs from the need to manage physical memory. Each process can swap out
memory when it needs to free up memory for other processes. It is a fundamental
abstraction that is useful for all operating systems.
\\
\linebreak
Copy-on-write paging is beneficial for both operating systems because it 
prevents unnecessary overhead. It anticipates common scenarios like the fork
and exec example for Linux in the similarities section and avoids 
unnecessarily copying the address space for a process for which its 
executable image will likely change soon after.
\\
\linebreak
Similar to the reason Linux provides two system calls for creating a process 
and then attaching an executable image to it, it sees memory mapping a file 
and assigning it as a shared memory segment as two different functionalities
that shouldn't be merged into one. This maintains modularity and ensures that
each function do only one thing. Windows implements shared memory as memory 
mapped files because they are related functionalities that can be combined 
into one. One process can create the memory mapping. Another process can
open the memory mapped file so that it is in its address space as well. 
\\
\linebreak
Windows makes the kernel's virtual memory part of every process's virtual 
memory so that kernel code can run in kernel mode without having to perform a 
context switch. Less of a process's virtual memory is private, but it is able
to run system calls much faster. This is important because performance is a 
primary concern for Windows systems. Linux is willing to sacrifice performance
for more private virtual address space for each process.
\section{I/O}
\subsection{Similarities}
Both Windows and Linux offer a number of similar system calls for I/O like 
open, read, write, ioctl, and close. Both operating systems view devices as 
files and control access to devices with basic file system functions. In 
Linux, these are open(), read(), write(), ioctl(), and close(). In Windows, 
I/O request packets (IRP) are created and then passed to the I/O subsystem. 
The IRPs specify the operations to be performed and that correspond to the 
I/O manager. These include CREATE, READ, WRITE, IOCTL, and CLOSE.
\\
\linebreak
Both Windows and Linux provide support for asynchronous I/O. It is a method of
allowing a process to continue execution after it submits an I/O request to 
not block until data becomes available. Linux provides this functionality 
through its AIO system calls like io\_setup, io\_destroy, io\_submit,
io\_cancel, and io\_getevents. Windows provides this functionality in a number
of ways and offers a number of options to implement it. Examples include 
creating an event object when making an I/O call and waiting on it at a later 
point in time, and specifying a queue to which a completion event will be 
added by the I/O manager.
\\
\linebreak
Both operating systems provide support for stacked device drivers. In Linux
they are called stacked modules. The idea is that drivers share certain 
functionalities and that multiple other drivers can be executed before 
executing the core functionality of a particular driver.
\\
\linebreak
Hibernation, or suspending to disk, is supported by both Linux and Windows.
It involves saving the computer's state to the disk and then powering down to
save power. Windows performs a number of optimizations like compression of 
memory pages and ignoring unmodified pages backed by disk.
\\
\linebreak
Both Windows and Linux support dynamically loadable modules. In Windows, these
modules are called dynamically linked libraries, or DLL's. In Linux, the 
modules are called shared objects or .so files. They allow for functions and 
data to be linked at runtime. 
\\
\linebreak
Both Windows and Linux have a uniform device access interface for their 
device drivers. In Windows, this is called the Windows Driver Model (WDM) 
that all device drivers must conform to. This exists so that application 
programs can access devices in a uniform way without having to worry about
the details of the device implementation. 
\subsection{Differences}
Windows offers support for a plug and play I/O subsystem while Linux does not.
The plug-and-play is a feature that provides automatic identification of 
devices and installation of device drivers. It sends requests to a number of
different buses like PCI, USB, and SATA to register its devices. Hardware 
resources are then allocated for the device. This allows for support many more
devices than Linux.
\\
\linebreak
DLL's and .so files are very different in the way that they link dynamic 
modules. When .so files are linked to a program, the references to all the 
module's functions and data are updated to point to the actual locations in 
memory where they are placed. DLL's use a look-up table to map references to 
the module's functions and data.
\subsection{Why Similarities and Differences Exist?}
Windows supports a complex plug and play I/O subsystem to support a large 
range of devices and to simplify device driver installation for the user. 
Linux does not provide this kind of plug and play subsystem because it is 
designed for skilled programmers who can install or even write their own 
drivers for devices. They need a simple, elegant interface to write their 
drivers and load them. Not as many devices are supported, but the understanding
is that Linux programmers would be able to install the device drivers on their
own.
%input the pygmentized output of mt19937ar.c, using a (hopefully) unique name
%this file only exists at compile time. Feel free to change that.
%\input{__mt19937ar.c.tex}
\nocite{*}
\RaggedRight
\bibliographystyle{plain-annote}
\bibliography{CS444_final_paper}
\end{document}
